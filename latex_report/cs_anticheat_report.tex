\documentclass{article}
\usepackage[utf8]{inputenc}
\usepackage{graphicx}
\usepackage{amsmath}
\usepackage{geometry}
\usepackage{hyperref}
\usepackage{booktabs}
\usepackage{float}

\geometry{a4paper, margin=1in}
\hypersetup{
    colorlinks=true,
    linkcolor=blue,
    filecolor=magenta,      
    urlcolor=cyan,
}

\title{A Statistical Approach to Anti-Cheat System Evaluation}
\author{Nasseem}
\date{\today}

\begin{document}

\maketitle

\begin{abstract}
Cheating in online multiplayer games undermines competitive integrity and degrades the player experience. This report details the development and comprehensive evaluation of a statistical anti-cheat system designed to identify cheaters based on in-game performance data. Using machine learning models trained on player statistics, the system flags suspicious behavior for further review. This document outlines the data processing pipeline, model architecture, and a multi-faceted evaluation including cross-validation, performance visualization, and a business impact analysis. The results demonstrate the system's potential as an effective screening tool and provide recommendations for its practical deployment.
\end{abstract}

\section{Introduction}
The proliferation of cheating software presents a significant challenge for the online gaming industry. Traditional anti-cheat methods, such as signature-based detection, are in a constant arms race with cheat developers. This project explores a complementary, data-driven approach. By analyzing aggregated player statistics, we can build machine learning models to identify players whose performance profiles are statistically anomalous and indicative of cheating.

The primary objective of this work is to evaluate the feasibility and performance of such a system. We aim to answer key questions regarding its accuracy, its reliability in distinguishing legitimate skill from cheating, and its practical business impact in a real-world operational scenario.

\section{Methodology}
The project follows a standard machine learning workflow, from data acquisition and feature engineering to model training and evaluation. The system is designed as an ensemble of models to provide a robust and reliable prediction score.

\subsection{Data Acquisition and Feature Engineering}
The dataset was sourced from a SQLite database (\texttt{anticheat.db}), containing player profiles and aggregated match statistics. The evaluation dataset consists of 917 players, of which 91 are confirmed cheaters (VAC banned) and 4 are flagged as statistical suspects, making for a cheater prevalence of 10.4\%.

\paragraph{Features} The following features were extracted for each player:
\begin{itemize}
    \item \texttt{avg\_kills}: Average kills per session.
    \item \texttt{avg\_deaths}: Average deaths per session.
    \item \texttt{avg\_headshots}: Average headshots per session.
    \item \texttt{avg\_accuracy}: Average accuracy per session.
    \item \texttt{max\_kills}: Maximum kills in a single session.
    \item \texttt{max\_accuracy}: Maximum accuracy in a single session.
    \item \texttt{total\_sessions}: Total number of recorded sessions.
\end{itemize}

Additional features, such as kill-death ratio (\texttt{kdr}) and headshot rate (\texttt{hs\_rate}), were engineered to better capture player performance dynamics.

\paragraph{Labeling}
Ground truth labels were established using a combination of known VAC (Valve Anti-Cheat) bans and statistical outlier detection, creating a dataset of "clean" players, "known cheaters", and "statistical suspects".

\subsection{Model Architecture}
The system utilizes an ensemble of machine learning models. The final prediction is an average of the probabilities from each individual model, which provides a more stable and reliable score than any single model. The models were trained on a balanced dataset to handle the inherent class imbalance between cheaters and legitimate players. The core components include:
\begin{itemize}
    \item \textbf{Preprocessor}: A standard scaler to normalize features.
    \item \textbf{Feature Selector}: A component to select the most informative features.
    \item \textbf{Models}: An ensemble of classifiers, including a Random Forest model (\texttt{rf\_balanced}) and a Gradient Boosting model (\texttt{gb\_balanced}).
\end{itemize}

\subsection{Evaluation Framework}
A comprehensive evaluation was conducted using a dedicated test dataset. The evaluation focuses on three key areas: cross-validation, performance visualization, and business impact.

\section{Results and Analysis}
The evaluation was performed using the \texttt{ComprehensiveEvaluator} class, which orchestrates the loading of data, execution of models, and generation of analytical results.

\subsection{Cross-Validation Performance}
Stratified K-Fold cross-validation was performed to assess the generalized performance of the models. The Area Under the ROC Curve (AUC) was used as the primary metric. The results are as follows:
\begin{itemize}
    \item \texttt{rf\_balanced}: AUC = 0.769 (+/- 0.087)
    \item \texttt{gb\_balanced}: AUC = 0.746 (+/- 0.109)
\end{itemize}

\subsection{Performance Visualizations}
Visualizations are critical for understanding model behavior. The following plots were generated to assess performance across different thresholds.

\paragraph{ROC Curves} The Receiver Operating Characteristic (ROC) curve (Figure \ref{fig:roc}) illustrates the trade-off between the True Positive Rate (TPR) and False Positive Rate (FPR). The high AUC score for the ensemble model indicates excellent discriminative ability.

\begin{figure}[H]
    \centering
    \includegraphics[width=0.8\textwidth]{evaluation_plots/roc_curves.png}
    \caption{ROC Curves for Individual Models and the Ensemble.}
    \label{fig:roc}
\end{figure}

\paragraph{Precision-Recall Curves} The Precision-Recall (PR) curve (Figure \ref{fig:pr}) is particularly informative for imbalanced datasets. It shows the trade-off between precision (the accuracy of positive predictions) and recall (the ability to find all positive samples).

\begin{figure}[H]
    \centering
    \includegraphics[width=0.8\textwidth]{evaluation_plots/precision_recall_curves.png}
    \caption{Precision-Recall Curves for Individual Models and the Ensemble.}
    \label{fig:pr}
\end{figure}

\paragraph{Feature Importance} Understanding which features the model relies on is key to trusting its decisions. The feature importance plot (Figure \ref{fig:importance}) from the Random Forest model shows that metrics like headshot ratio, average accuracy, and kill-death ratio are the most significant predictors.

\begin{figure}[H]
    \centering
    \includegraphics[width=0.8\textwidth]{evaluation_plots/feature_importance.png}
    \caption{Feature Importance from the Random Forest Model.}
    \label{fig:importance}
\end{figure}

\subsection{Business Impact Analysis}
Beyond theoretical metrics, it is crucial to understand the real-world impact of deploying the system. We analyzed the effect of different detection thresholds on key business metrics. Table \ref{tab:business} summarizes the trade-offs.

\begin{table}[H]
    \centering
    \caption{Business Impact Analysis at Different Thresholds}
    \label{tab:business}
    \begin{tabular}{@{}lccccccc@{}}
    \toprule
    Threshold & Flagged & Caught & False+ & Missed & Recall & Precision & FPR \\
    \midrule
    0.1 & 525 & 92 & 433 & 3 & 0.968 & 0.175 & 0.527 \\
    0.2 & 428 & 91 & 337 & 4 & 0.958 & 0.213 & 0.410 \\
    0.3 & 325 & 89 & 236 & 6 & 0.937 & 0.274 & 0.287 \\
    0.4 & 231 & 88 & 143 & 7 & 0.926 & 0.381 & 0.174 \\
    0.5 & 145 & 87 & 58 & 8 & 0.916 & 0.600 & 0.071 \\
    0.6 & 120 & 85 & 35 & 10 & 0.895 & 0.708 & 0.043 \\
    0.7 & 91 & 72 & 19 & 23 & 0.758 & 0.791 & 0.023 \\
    0.8 & 45 & 41 & 4 & 54 & 0.432 & 0.911 & 0.005 \\
    0.9 & 24 & 23 & 1 & 72 & 0.242 & 0.958 & 0.001 \\
    \bottomrule
    \end{tabular}
\end{table}

Based on this analysis, we can define several operating points:
\begin{itemize}
    \item \textbf{Conservative (High Precision):} A threshold of 0.5 flags 145 players and catches 87 cheaters, with a precision of 0.600 and recall of 0.916.
    \item \textbf{Balanced (High F1-Score):} A threshold of 0.6 flags 120 players and catches 85 cheaters, with a precision of 0.708 and recall of 0.895.
\end{itemize}

\section{Conclusion and Recommendations}
The statistical anti-cheat system demonstrates strong potential as a tool for identifying cheaters. The ensemble model is robust and achieves high performance in terms of both AUC and Average Precision.

Based on the evaluation, we provide the following recommendations:
\begin{enumerate}
    \item \textbf{Use as a Screening Tool:} The system is best suited for initial screening. Due to the non-zero false positive rate, flagged players should be subject to further human review before any punitive action, such as a ban, is taken.
    \item \textbf{Implement a Tiered Strategy:} Use different thresholds for different purposes. A high-precision threshold can trigger high-priority reviews, while a balanced threshold can be used for broader monitoring.
    \item \textbf{Future Work:} Performance could be further improved by incorporating temporal analysis (how player stats change over time) and analyzing more granular data, such as in-match positional data or action sequences.
\end{enumerate}

This project successfully establishes a framework for a data-driven anti-cheat system, providing a valuable layer of defense against cheating in online games.

\end{document}
